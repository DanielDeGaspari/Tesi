	%***********************************%
	%									%
	%		Tesi di Laurea Triennale	%
	%		  	di Daniel De Gaspari	%
	%									%
	%				 7 Dicembre 2017	%
	%									%
	%***********************************%
	
\documentclass[ 10pt,
				a4paper,										% formato A4
				twoside,										% impagina per fronte-retro
				openright,										% inizio capitoli a destra
				] {book}

%******************************************
% Importazione packages
%******************************************
\usepackage[utf8]{inputenc}										% codifica di input
\usepackage{graphicx}											% immagini
\usepackage{chngpage, calc}										% centra il frontespizio
\usepackage[binding=5mm]{layaureo}								% margini ottimizzati per l'A4; rilegatura di 5mm
\usepackage{microtype}											% microtipografia
\usepackage{mparhack, fixltx2e, relsize}						% finezze tipografiche
\usepackage{hyperref}											% collegamenti ipertestuali
\usepackage{bookmark}											% segnalibri
\usepackage{caption}											% didascalie
\usepackage{csquotes}											% gestisce automaticamente i caratteri (")
\usepackage{emptypage}											% pagine vuote senza intestazione e pie di pagina
\usepackage{nameref}											% visualizza i nomi dei riferimenti
\usepackage[font=small]{quoting}								% citazioni
\usepackage[italian]{varioref}									% riferimenti completi alla pagina
\usepackage[dvpsnames]{xcolor}									% colori

\usepackage[toc, acronym]{glossaries}							% glossario

%%% ****************************************** %%%
%	File contenente impostazioni della tesi	     %
%%% ****************************************** %%%

%%% ****************************************** %%%
%	Informazioni generali						 %
%%% ****************************************** %%%

% ******************************************
%	Autore
% ******************************************
\newcommand{\autore}{Daniel De Gaspari}

% ******************************************
%	Tesi
% ******************************************
\newcommand{\titoloTesi}{Titolo della tesi}
\newcommand{\tipologiaTesi}{Tesi di laurea triennale}

% ******************************************
%	Univeristà
% ******************************************
\newcommand{\universita}{Università degli Studi di Padova}
\newcommand{\cdl}{Corso di Laurea in Informatica}
\newcommand{\dipartimento}{Dipartimento di Matematica "Tullio Levi-Civita"}
\newcommand{\luogo}{Padova}
\newcommand{\myAA}{2017/2018}
\newcommand{\dataDiscussione}{Dicembre 2017}

% ******************************************
%	Relatore
% ******************************************
\newcommand{\statusrelatore}{Prof.}
\newcommand{\relatore}{Tullio Vardanega}

% ******************************************
%	Azienda
% ******************************************
\newcommand{\nomeAzienda}{InfoCamere S.C.p.A. }
\newcommand{\locazioneAzienda}{Padova (PD)}


%%% ****************************************** %%%
%	Impostazioni di graphicx					 %
%%% ****************************************** %%%
\graphicspath{{images/}}									% path a cartella contenente le immagini


%%% ****************************************** %%%
%	Impostazioni di glossaries					 %
%%% ****************************************** %%%
%******************************************
% Acronimi
%******************************************
\renewcommand{\acronymname}{Acronimi e abbreviazioni}

%******************************************
% Glossario
%******************************************
\newglossaryentry{test}
{
	name=\glslink{test}{TEST},
	text=test,
	sort=test,
	description={testo test}
} 										% DataBase dei termini
\makeglossaries
\newcommand{\glsfirstoccur}{\appendix{{[g]}}}


%%% ****************************************** %%%
%	Impostazioni di hyperref					 %
%%% ****************************************** %%%
\hypersetup{
	colorlinks = true,
	linktocpage = true,
	pdfstartpage = 1,
	pdfstartview = FitV,
	breaklinks = true,
	pdfpagemode = UseNone,
	pageanchor = true,
	pdfpagemode = UseOutlines,
	plainpages = false,
	bookmarksnumbered,
	bookmarksopen = true,
	bookmarksopenlevel = 1,
	hypertexnames = true,
	pdfhighlight = /0,
	urlcolor = webbrown,
	linkcolor = blue,
	citecolor = webgreen,
	pdftitle={\titoloTesi}
	pdfauthor={\textcopyright\ \autore, \universita, \cdl},
	pdfsubject={},
	pdfkeywords={},
	pdfcreator={pdfLaTeX},
	pdfproducer={LaTeX}
}


%%% ****************************************** %%%
%	Impostazioni di xcolor						 %
%%% ****************************************** %%%
\definecolor{webgreen}{rgb}{0,.5,0}
\definecolor{webbrown}{rgb}{.6,0,0}
\definecolor{blue}{cmyk}{0.71,0.53,0.00,0.12}									% file di configurazione contenente impostazioni personali

%******************************************
% Documento
%******************************************

\begin{document}
	%******************************************
	% Parte iniziale
	%******************************************
	\frontmatter
	% !TEX root = ../Tesi.tex

%******************************************
% Frontespizio
%******************************************

\begin{titlepage}
	
	\begin{center}
		
		\begin{LARGE}
			\textbf{\universita}\\
		\end{LARGE}

		\vspace{10pt}

		\begin{Large}
			\textsc{\dipartimento}\\
		\end{Large}

		\vspace{10pt}
		
		\begin{large}
			\textsc{\cdl}\\
		\end{large}

		\vspace{30pt}
		
		\begin{figure}[htbp]
			\begin{center}
				\includegraphics[height=6cm]{logo-unipd}
			\end{center}
		\end{figure}
	
		\vspace{30pt}
		
		%TODO cambiare titolo tesi
		\begin{LARGE}
			\begin{center}
				\textbf{\titoloTesi}\\
			\end{center}
		\end{LARGE}

		\vspace{10pt}

		\begin{large}
			\textsl{\tipologiaTesi}\\
		\end{large}
	
		\vspace{40pt}
		
		\begin{large}
			\begin{flushleft}
				\textit{Relatore}\\
				\vspace{5pt}
				\statusrelatore \relatore
			\end{flushleft}
		
			\vspace{0pt}
		
			\begin{flushright}
				\textit{Laureando}\\
				\vspace{5pt}
				\autore
			\end{flushright}
		
		\end{large}
	
		\vspace{40pt}
		\line(1,0){338} \\

		%TODO controllare A.A.		
		\begin{normalsize}
			\textsc{Anno Accademico \myAA}
		\end{normalsize}

	\end{center}

\end{titlepage}
	% !TEX root = ../Tesi.tex

%******************************************
% Colophon
%******************************************

\clearpage
\phantomsection
\thispagestyle{empty}

\hfill

\vfill

\noindent\autore: \textit{\titoloTesi,}
\tipologiaTesi,
\textcopyright\ \dataDiscussione.

	% !TEX root = ../Tesi.tex

%******************************************
% Dedica/Citazione
%******************************************

\cleardoublepage
\phantomsection
\thispagestyle{empty}
\pdfbookmark{Dedica}{Dedica}

\vspace*{3cm}

%TODO cambiare citazione
\begin{center}
	C'è una forza motrice più forte del vapore, dell'elettricità e dell'energia atomica: \textbf{la volontà}. \\ \medskip
	--- Albert Einstein
\end{center}
\medskip
	% !TEX root = ../Tesi.tex

%******************************************
% Sommario
%******************************************
\cleardoublepage
\phantomsection
\pdfbookmark{Sommario}{Sommario}
\begingroup
\let\clearpage\relax
\let\cleardoublepage\relax
\let\cleardoublepage\relax

\chapter*{Sommario}

Il presente documento descrive il lavoro svolto durante il periodo di stage dal laureando \autore, della durata di circa trecento ore, presso l'azienda \nomeAzienda di \locazioneAzienda. \\
Gli obiettivi da raggiungere erano molteplici. \\
Lo scopo dello stage era quello di analizzare le caratteristiche dei motori di ricerca \gls{open source} nell'ambito dei siti web di tipo informativo.\\
In primo luogo era richiesto un approfondimento delle caratteristiche istituzionali dei siti web delle Camere di Commercio. \\
Successivamente, l'azienda richiedeva di analizzare le potenzialità e specificità di motori di ricerca \gls{Solr} e \gls{ElasticSearch}. \\
Il passo successivo consisteva nel realizzare un prototipo di un sito web in tecnologia \gls{Drupal}, con i due motori di ricerca precedentemente citati. \\
Infine, era richiesta una relazione finale delle potenzialità emerse nell'utilizzo dei due motori di ricerca. \\
I primi due capitoli del presente documento hanno lo scopo di presentare il contesto aziendale in cui è stato sostenuto lo stage e di spiegare come il progetto di stage si renda utile all’interno della strategia aziendale. Il terzo capitolo documenta invece lo svolgimento dello stage descrivendo le attività che sono state portate a termine, i punti salienti del progetto stesso e le principali scelte attuate. Il quarto ed ultimo capitolo presenta infine una valutazione dello svolgimento dello stage rispetto agli obiettivi aziendali e alle conoscenze acquisite dallo studente.


\endgroup

\vfill
	
	\cleardoublepage
	%******************************************
	% Parte centrale
	%******************************************
	\mainmatter

	%******************************************
	% Parte finale
	%******************************************
	\backmatter
	\printglossaries
	
\end{document}