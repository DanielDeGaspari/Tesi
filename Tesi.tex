	%***********************************%
	%									%
	%		Tesi di Laurea Triennale	%
	%		  	di Daniel De Gaspari	%
	%									%
	%				 7 Dicembre 2017	%
	%									%
	%***********************************%
	
% !TEX encoding = UTF-8
% !TEX TS-program = pdflatex
% !TEX spellcheck = it-IT
	
\documentclass[ 10pt,
				a4paper,										% formato A4
				twoside,										% impagina per fronte-retro
				openright,										% inizio capitoli a destra
				english,
				italian,
				] {book}

%******************************************
% Importazione packages
%******************************************
\usepackage[utf8]{inputenc}										% codifica di input
\usepackage[english, italian]{babel}							% lingua inglese e italiana
\usepackage{graphicx}											% immagini
\usepackage{chngpage, calc}										% centra il frontespizio
\usepackage[binding=5mm]{layaureo}								% margini ottimizzati per l'A4; rilegatura di 5mm
\usepackage{microtype}											% microtipografia
\usepackage{mparhack, fixltx2e, relsize}						% finezze tipografiche
\usepackage{hyperref}											% collegamenti ipertestuali
\usepackage{bookmark}											% segnalibri
\usepackage{caption}											% didascalie
\usepackage{csquotes}											% gestisce automaticamente i caratteri (")
\usepackage{emptypage}											% pagine vuote senza intestazione e pie di pagina
\usepackage{nameref}											% visualizza i nomi dei riferimenti
\usepackage[font=small]{quoting}								% citazioni
\usepackage[italian]{varioref}									% riferimenti completi alla pagina
\usepackage[dvpsnames]{xcolor}									% colori

\usepackage[toc, acronym]{glossaries}							% glossario

\usepackage[backend=biber,style=verbose-ibid,hyperref,backref]{biblatex}	
% eccellente pacchetto per la bibliografia; 
% produce uno stile di citazione autore-anno; 
% lo stile "numeric-comp" produce riferimenti numerici
% per includerlo nel documento bisogna:
% 1. compilare una prima volta tesi.tex;
% 2. eseguire: biber tesi
% 3. compilare ancora tesi.tex.

%%% ****************************************** %%%
%	File contenente impostazioni della tesi	     %
%%% ****************************************** %%%

%%% ****************************************** %%%
%	Informazioni generali						 %
%%% ****************************************** %%%

% ******************************************
%	Autore
% ******************************************
\newcommand{\autore}{Daniel De Gaspari}

% ******************************************
%	Tesi
% ******************************************
\newcommand{\titoloTesi}{Titolo della tesi}
\newcommand{\tipologiaTesi}{Tesi di laurea triennale}

% ******************************************
%	Univeristà
% ******************************************
\newcommand{\universita}{Università degli Studi di Padova}
\newcommand{\cdl}{Corso di Laurea in Informatica}
\newcommand{\dipartimento}{Dipartimento di Matematica "Tullio Levi-Civita"}
\newcommand{\luogo}{Padova}
\newcommand{\myAA}{2017/2018}
\newcommand{\dataDiscussione}{Dicembre 2017}

% ******************************************
%	Relatore
% ******************************************
\newcommand{\statusrelatore}{Prof.}
\newcommand{\relatore}{Tullio Vardanega}

% ******************************************
%	Azienda
% ******************************************
\newcommand{\nomeAzienda}{InfoCamere S.C.p.A. }
\newcommand{\locazioneAzienda}{Padova (PD)}


%%% ****************************************** %%%
%	Impostazioni di graphicx					 %
%%% ****************************************** %%%
\graphicspath{{images/}}									% path a cartella contenente le immagini


%%% ****************************************** %%%
%	Impostazioni di glossaries					 %
%%% ****************************************** %%%
%******************************************
% Acronimi
%******************************************
\renewcommand{\acronymname}{Acronimi e abbreviazioni}

%******************************************
% Glossario
%******************************************
\newglossaryentry{test}
{
	name=\glslink{test}{TEST},
	text=test,
	sort=test,
	description={testo test}
} 										% DataBase dei termini
\makeglossaries
\newcommand{\glsfirstoccur}{\appendix{{[g]}}}


%%% ****************************************** %%%
%	Impostazioni di hyperref					 %
%%% ****************************************** %%%
\hypersetup{
	colorlinks = true,
	linktocpage = true,
	pdfstartpage = 1,
	pdfstartview = FitV,
	breaklinks = true,
	pdfpagemode = UseNone,
	pageanchor = true,
	pdfpagemode = UseOutlines,
	plainpages = false,
	bookmarksnumbered,
	bookmarksopen = true,
	bookmarksopenlevel = 1,
	hypertexnames = true,
	pdfhighlight = /0,
	urlcolor = webbrown,
	linkcolor = blue,
	citecolor = webgreen,
	pdftitle={\titoloTesi}
	pdfauthor={\textcopyright\ \autore, \universita, \cdl},
	pdfsubject={},
	pdfkeywords={},
	pdfcreator={pdfLaTeX},
	pdfproducer={LaTeX}
}


%%% ****************************************** %%%
%	Impostazioni di xcolor						 %
%%% ****************************************** %%%
\definecolor{webgreen}{rgb}{0,.5,0}
\definecolor{webbrown}{rgb}{.6,0,0}
\definecolor{blue}{cmyk}{0.71,0.53,0.00,0.12}									% file di configurazione contenente impostazioni personali

%******************************************
% Documento
%******************************************

\begin{document}
	%******************************************
	% Parte iniziale
	%******************************************
	\frontmatter
	% !TEX root = ../Tesi.tex

%******************************************
% Frontespizio
%******************************************

\begin{titlepage}
	
	\begin{center}
		
		\begin{LARGE}
			\textbf{\universita}\\
		\end{LARGE}

		\vspace{10pt}

		\begin{Large}
			\textsc{\dipartimento}\\
		\end{Large}

		\vspace{10pt}
		
		\begin{large}
			\textsc{\cdl}\\
		\end{large}

		\vspace{30pt}
		
		\begin{figure}[htbp]
			\begin{center}
				\includegraphics[height=6cm]{logo-unipd}
			\end{center}
		\end{figure}
	
		\vspace{30pt}
		
		%TODO cambiare titolo tesi
		\begin{LARGE}
			\begin{center}
				\textbf{\titoloTesi}\\
			\end{center}
		\end{LARGE}

		\vspace{10pt}

		\begin{large}
			\textsl{\tipologiaTesi}\\
		\end{large}
	
		\vspace{40pt}
		
		\begin{large}
			\begin{flushleft}
				\textit{Relatore}\\
				\vspace{5pt}
				\statusrelatore \relatore
			\end{flushleft}
		
			\vspace{0pt}
		
			\begin{flushright}
				\textit{Laureando}\\
				\vspace{5pt}
				\autore
			\end{flushright}
		
		\end{large}
	
		\vspace{40pt}
		\line(1,0){338} \\

		%TODO controllare A.A.		
		\begin{normalsize}
			\textsc{Anno Accademico \myAA}
		\end{normalsize}

	\end{center}

\end{titlepage}
	% !TEX root = ../Tesi.tex

%******************************************
% Colophon
%******************************************

\clearpage
\phantomsection
\thispagestyle{empty}

\hfill

\vfill

\noindent\autore: \textit{\titoloTesi,}
\tipologiaTesi,
\textcopyright\ \dataDiscussione.

	% !TEX root = ../Tesi.tex

%******************************************
% Dedica/Citazione
%******************************************

\cleardoublepage
\phantomsection
\thispagestyle{empty}
\pdfbookmark{Dedica}{Dedica}

\vspace*{3cm}

%TODO cambiare citazione
\begin{center}
	C'è una forza motrice più forte del vapore, dell'elettricità e dell'energia atomica: \textbf{la volontà}. \\ \medskip
	--- Albert Einstein
\end{center}
\medskip
	% !TEX root = ../Tesi.tex

%******************************************
% Sommario
%******************************************
\cleardoublepage
\phantomsection
\pdfbookmark{Sommario}{Sommario}
\begingroup
\let\clearpage\relax
\let\cleardoublepage\relax
\let\cleardoublepage\relax

\chapter*{Sommario}

Il presente documento descrive il lavoro svolto durante il periodo di stage dal laureando \autore, della durata di circa trecento ore, presso l'azienda \nomeAzienda di \locazioneAzienda. \\
Gli obiettivi da raggiungere erano molteplici. \\
Lo scopo dello stage era quello di analizzare le caratteristiche dei motori di ricerca \gls{open source} nell'ambito dei siti web di tipo informativo.\\
In primo luogo era richiesto un approfondimento delle caratteristiche istituzionali dei siti web delle Camere di Commercio. \\
Successivamente, l'azienda richiedeva di analizzare le potenzialità e specificità di motori di ricerca \gls{Solr} e \gls{ElasticSearch}. \\
Il passo successivo consisteva nel realizzare un prototipo di un sito web in tecnologia \gls{Drupal}, con i due motori di ricerca precedentemente citati. \\
Infine, era richiesta una relazione finale delle potenzialità emerse nell'utilizzo dei due motori di ricerca. \\
I primi due capitoli del presente documento hanno lo scopo di presentare il contesto aziendale in cui è stato sostenuto lo stage e di spiegare come il progetto di stage si renda utile all’interno della strategia aziendale. Il terzo capitolo documenta invece lo svolgimento dello stage descrivendo le attività che sono state portate a termine, i punti salienti del progetto stesso e le principali scelte attuate. Il quarto ed ultimo capitolo presenta infine una valutazione dello svolgimento dello stage rispetto agli obiettivi aziendali e alle conoscenze acquisite dallo studente.


\endgroup

\vfill
	% !TEX root = ../Tesi.tex

%******************************************
% Ringraziamenti
%******************************************
\cleardoublepage
\phantomsection
\pdfbookmark{ringraziamenti}{ringraziamenti}

%TODO modificare citazione
\begin{flushright} {
	\slshape
	"Citazione"
} \\
	\medskip
		--- Confucius
\end{flushright}

\begingroup
\let\clearpage\relax
\let\cleardoublepage\relax

%TODO modificare ringraziamenti
\chapter*{Ringraziamenti}

\noindent \textit{Ringraziamenti}\\

\bigskip

\noindent \textit{\luogo, \dataDiscussione}
\hfill \autore

\endgroup
	% !TEX root = ../Tesi.tex

%******************************************
% Indice
%******************************************
\cleardoublepage
\pdfbookmark{\contentsname}{tableofcontents}
\setcounter{tocdepth}{2}
\tableofcontents
\clearpage

\begingroup

	\let\clearpage\relax
	\let\cleardoublepage\relax
	\let\cleardoublepage\relax
	
	%******************************************
	% Elenco delle figure
	%******************************************
	\phantomsection
	\pdfbookmark{\listtablename}{lot}
	\listoftables
	
	\vspace*{8ex}

\endgroup

\cleardoublepage
	
	\cleardoublepage
	%******************************************
	% Parte centrale
	%******************************************
	\mainmatter
	% !TEX root = ../../Tesi.tex

%******************************************
% L'azienda
%******************************************

\chapter{L'azienda}
\label{cap:azienda}

\section{Il Profilo Aziendale}
\label{sec:il_profilo_aziendale}

	\subsection{Le origini: Cerved}
	Nata inizialmente come Cerved (Centro Regionale Veneto Elaborazione Dati), \nomeAzienda è stata fondata nel Dicembre del 1974 a Padova dal Professor Mario Volpato, allora Presidente della Camera di Commercio di Padova e Professore di Calcolo delle probabilità all'Università di Padova. \\
	L'obiettivo era di raccogliere e conservare i dati ufficiali anagrafici e amministrativi delle imprese della provincia di Padova in un modo nuovo rispetto a quanto previsto fino ad allora: la conservazione di quei dati su un registro cartaceo, come si faceva dal medioevo ai tempi delle comunità dei mercanti, non bastava più a garantire l'efficienza del mercato e a stimolare lo sviluppo economico. \\
	Le prime tecnologie informatiche aprivano nuovi orizzonti al trattamento massivo e veloce dei dati. Evolveva rapidamente la concezione di una gestione intelligente delle notizie amministrative sulla vita delle imprese, per trasformarle in informazioni rielaborabili ed utilizzabili in modi nuovi da tutti. Nasceva l’idea di valorizzare i dati ufficiali forniti dalle imprese, restituendoli al mercato e alle imprese stesse come informazioni utili per accrescere la propria competitività e progettare lo sviluppo.\\
	Si gettava il seme dell’efficienza nell’organizzazione delle Camere di Commercio, una base nuova per costruire un patto trasparente e vantaggioso tra imprese e Pubblica Amministrazione.
	
	\subsection{Anni '90: InfoCamere}
	All'inizio degli anni '90 aumentava sempre più la competizione globale e le sfide per portare l'Italia nella modernità. A tal fine, nel 1993, venne emanata una riforma (legge 29 Dicembre 1993, n. 580) che attribuiva alle Camere un'autonomia rispetto al governo centrale, mediante attribuzione della potestà statutaria e di autonomia finanziaria, oltre al riconoscimento del ruolo finalizzato alla pubblicizzazione delle imprese. Le Camere di Commercio Italiane hanno così modo di vedere un profondo rinnovamento in vari ambiti e in particolar modo nell'ambito tecnologico. \\
	Nel 1995, per scissione da Cerved, nasce InfoCamere che raccoglie la sfida di realizzare il Registro delle imprese. Previsto dal codice civile fin dal 1942 e mai attuato, in due anni, con uno di anticipo sulle previsioni, il risultato è raggiunto: prende vita il primo esempio in Europa di registro pubblico sulle imprese  totalmente telematico; assieme ad un ecosistema di servizi sviluppati attorno al Registro delle imprese, è stato possibile semplificare i processi tra le imprese stesse e la Pubblica Amministrazione. \\
	A quella sfida ne seguono altre che rispondono ai nomi di ‘firma digitale’, ‘posta elettronica certificata’, ‘comunicazione unica’, ecc... . \\
	Attraverso InfoCamere, servizi e tecnologie digitali di frontiera diventavano patrimonio quotidiano della comunità delle imprese e dei professionisti, influendo sulle abitudini di lavoro di migliaia di italiani e stimolando i processi di innovazione nella Pubblica Amministrazione.
	
	\begin{figure}[htbp]
		\begin{center}
			\includegraphics[height=3cm]{logo-infocamere}
			\captionof{figure}{Logo infocamere}
		\end{center}
	\end{figure}

	\subsection{Servizi offerti}
	\nomeAzienda è la società consortile di informatica delle Camere di Commercio Italiane. Ha realizzato e gestisce il sistema telematico nazionale che collega tra loro tutte le Camere di Commercio, oltre alle rispettive sedi distaccate. \\
	Sua funzione istituzionale è anche la gestione e divulgazione del patrimonio informativo camerale, con particolare riferimento alle informazioni derivanti dal Registro delle imprese.\\
	Le banche dati camerali sono rese disponibili direttamente a imprese, pubbliche amministrazioni, professionisti e cittadini tramite il portale delle Camere di Commercio. \\
	La Società fornisce alle pubbliche amministrazioni l'accesso al Registro Imprese, assicurando loro l'accessibilità dei dati senza oneri, salvo quelli per la fornitura telematica e i servizi a valore aggiunto. \\
	Tramite il sito del Registro delle imprese si può accedere agli strumenti per lo svolgimento delle pratiche telematiche, tra cui la Comunicazione Unica per l'attività d'impresa, valida anche per Agenzia delle Entrate, INPS, INAIL e Albo Artigiani. Il Registro Imprese è inoltre uno strumento di trasparenza amministrativa che fornisce un contributo importante nella lotta contro la criminalità economica. L'azienda ha infatti sviluppato per le autorità investigative alcuni servizi che, in questa direzione, consentono analisi mirate per monitorare fenomeni anomali. \\
	InfoCamere ha realizzato, per conto delle Camere di Commercio, l'infrastruttura tecnologica che garantisce il corretto funzionamento degli Sportelli Unici per le Attività Produttive (SUAP). \\
	Tra le realizzazioni di InfoCamere per il Sistema camerale vi è anche la procedura informatica che consente di gestire il servizio di conciliazione online (Concilia Camera), fornendo così ad imprese, consumatori e professionisti uno strumento che permette di ricevere assistenza specializzata nel raggiungimento di un accordo per risolvere in modo semplice, rapido, economico e sicuro una controversia, evitando di ricorrere alla giustizia ordinaria. \\
	InfoCamere è inoltre l'Autorità di Certificazione Nazionale che rilascia i certificati digitali delle Carte Tachigrafiche.
	La società si è dotata di un Sistema di Gestione della Sicurezza delle Informazioni certificato secondo lo standard ISO/IEC 27001, avendo conseguito nel 2012 la prima certificazione di conformità ISO/IEC 27001:2005 e a Marzo 2015 la ricertificazione secondo la nuova versione ISO/IEC 27001: 2013.

\section{Organizzazione aziendale}
\label{sec:organizzazione_aziendale}
Questa sezione conterrà l'organizzazione dell'azienda; essendo un'azienda molto strutturata e con molti dipendenti, l'organizzazione aziendale si concentrerà sul contesto in cui ho operato.

\section{Processi aziendali}
\label{sec:processi_aziendali}
Questa sezione parlerà dei processi interni aziendali; essendo un'azienda molto strutturata e con molti dipendenti, i processi descritti si concentreranno sul contesto in cui ho operato.

\section{Tecnologie utilizzate}
\label{sec:tecnologie_utilizzate}
Questa sezione presenterà le tecnologie utilizzate dall'azienda, ristrette all'ambito in cui ho operato.

\section{Rapporto con l'innovazione}
\label{sec:rapporto_con_innovazione}
Questa sezione presenterà il rapporto dell'azienda con l'avanzamento tecnologico.
		% L'azienda
	% !TEX root = ../../Tesi.tex

%******************************************
%	Lo stage
%******************************************

\chapter{Lo stage}
\label{cap:stage}

\section{Gli stage in azienda}
Questa sezione presenterà le motivazioni per cui l'azienda ospita stage.

\section{L'offerta di stage}

	\subsection{Presentazione del progetto}
	Qui verrà presentato il progetto di stage.
	
	\subsection{Aspettative aziendali}
	Qui verranno presentate le aspettative aziendali legate all'offerta di stage.
	
	\subsection{Vincoli}
	Qui verranno presentati i vincoli legati allo stage.

\section{Vantaggi personali}
In questa sezione descriverò le aspettative personali legate all'accettazione dell'offerta di stage.
		% Lo stage
	% !TEX root = ../../Tesi.tex

%******************************************
%	Resoconto dello stage
%******************************************

\chapter{Resoconto dello stage}
\label{cap:resoconto_dello_stage}

	\section{Individuazione dei motori di ricerca}
	Ad oggi, sono disponibili in rete un gran numero di motori di ricerca, sia proprietari, sia \gls{open source}. Questi, mettono a disposizione funzionalità di ricerca più o meno avanzate e, a seconda di vari fattori come possono essere, ad esempio, la tipologia e la quantità di dati che vogliamo indicizzare, rendendoli ricercabili agli utenti, possiamo optare per l'uno piuttosto che per l'altro. \\
	La scelta del motore di ricerca più adatto alle specifiche esigenze, non è questione affatto banale; molte volte esistono varie soluzioni in grado di modellare bene il problema che dobbiamo affrontare, simili tra loro, che differiscono però per alcuni aspetti chiave. \\
	Focalizzando l'attenzione sulle esigenze dettate dai siti informativi Camerali, possiamo già applicare un primo filtro sui motori di ricerca da utilizzare: motori di ricerca proprietari \textit{vs} motori di ricerca \gls{open source}. \\
	Un motore di ricerca proprietario rappresenta indubbiamente un onere per l'azienda e, oltre a ciò, quest'ultima non ha il pieno controllo sulla destinazione finale dei dati; queste questioni non si hanno invece in un sistema \gls{open source}, nel quale il codice sorgente è liberamente modificabile e analizzabile, avendo dunque il pieno controllo sull'intero sistema. \\
	Nella figura 3.1 è presente una classifica dei motori di ricerca, aggiornata a Novembre 2017, calcolata su parametri che rappresentano la popolarità del sito (maggiori informazioni sui punteggi assegnati sono disponili al seguente indirizzo: \url{https://db-engines.com/en/ranking_definition}). Da questa tabella, notiamo come in testa alla classifica siano presenti tre motori di ricerca che hanno un ampio margine di distacco, in termini di punteggio ad essi attribuito, rispetto alle tecnologie che li precedono. Di questi, solamente i primi due (\gls{Solr} e \gls{ElasticSearch}) sono alternative \gls{open source}. \\
	Oltre a ciò, per entrambi i motori di ricerca appena citati, sono disponibili \glspl{Modulo} \gls{Drupal} che rendono possibile la semplice integrazione tra i motori di ricerca e il \gls{CMSg} di interesse. \\
	Le motivazioni appena descritte, hanno così portato alla selezione di \gls{Solr} e \gls{ElasticSearch} come motori di ricerca da studiare nell'ambito dello stage.

	\begin{figure}[htbp]
		\begin{center}
			\includegraphics[width=13cm]{tabella_SE}
			\captionof{figure}{\cite{site:confronto_SE}}
		\end{center}
	\end{figure}
	
	\section{Pianificazione}
	In questa sezione presenterò come è avvenuta la pianificazione dello stage.
	%Ambiente di sviluppo in locale
	
	\section{I siti istituzionali delle Camere di Commercio}

		\subsection{Funzionalità di ricerca attuali}
		Questa sezione presenterà la struttura dei siti camerali attualmente in produzione, ponendo l'accento sugli strumenti messi a disposizione all'utente per ritrovare i contenuti in esso presenti.
		
		\subsection{Possibile evoluzione}
		Questa sezione presenterà una possibile evoluzione dei siti camerali, contenente funzionalità di ricerca attualmente non presenti.

	\section{Ricerca nativa Drupal}

		\subsection{Introduzione a Drupal}
		Qui verrà introdotto Drupal, spiegando cos'è e come funziona.
		
		\subsection{Ricerca di base e avanzata}
		Qui verranno presentate le principali funzionalità offerte dalla prima tipologia di ricerca nativa Drupal.
		
		\subsection{Ricerca con Search API}
		Qui verranno presentate le principali funzionalità offerte dalla seconda tipologia di ricerca nativa Drupal.
		
		\subsection{Considerazioni di Drupal nativo}
		Questa sezione conterrà conclusioni riguardanti le funzionalità di ricerca offerte globalmente dalla ricerca nativa Drupal.

	\section{Ricerca con Solr}

		\subsection{Introduzione a Solr}
		Qui verrà introdotto Solr, spiegando cos'è e come funziona.
		
		\subsection{Principali funzionalità di ricerca}
		Qui verranno presentate le principali funzionalità di ricerca offerte dal motore di ricerca Solr, di possibile interesse per i siti camerali.
			
		\subsection{Integrazione con Drupal}
		Qui verranno discusse le funzionalità di ricerca derivanti dall'integrazione tra Solr e Drupal.
			
			\subsubsection{Apache Solr}
			Qui verranno discusse le funzionalità di ricerca derivanti dall'integrazione tra Solr e Drupal mediante il modulo Apache Solr Search.
			
			\subsubsection{Search API Solr}
			Qui verranno discusse le funzionalità di ricerca derivanti dall'integrazione tra Solr e Drupal mediante il modulo Search API Solr Search.
	
	\section{Ricerca con ElasticSearch}
	
		\subsection{Introduzione a ElasticSearch}
		Qui verrà introdotto ElasticSearch, spiegando cos'è e come funziona.

		\subsection{Principali funzionalità di ricerca}
		Qui verranno presentate le principali funzionalità di ricerca offerte dal motore di ricerca ElasticSearch, di possibile interesse per i siti camerali.
		
		\subsection{Integrazione con Drupal}
		Qui verranno discusse le funzionalità di ricerca derivanti dall'integrazione tra ElasticSearch e Drupal.
		%Accenno alla versione Drupal8?
		
			\subsubsection{Search API ElasticSearch}
			Qui verranno discusse le funzionalità di ricerca derivanti dall'integrazione tra ElasticSearch e Drupal mediante il modulo Search API ElasticSearch.
			
	\section{Considerazioni finali sui motori di ricerca esaminati}
	Questa sezione conterrà un confronto tra le principali funzionalità, possibilmente di interesse per l'azienda, offerte dalle tecnologie esaminate e quale di queste potrebbe essere la più adatta ai siti camerali.		% Resoconto dello stage
	% !TEX root = ../../Tesi.tex

%******************************************
%	Valutazione retrospettiva
%******************************************

\chapter{Valutazione retrospettiva}
\label{valutazione_retrospettiva}

\section{Bilancio degli obiettivi}

	\subsection{Aziendali}
	
	Riprendendo gli obiettivi posti inizialmente dall'azienda riguardanti lo stage, presentati nella \hyperref[sub:obiettivi_posti_azienda]{sezione 2.2.2 : Obiettivi posti dall'azienda}, posso affermare di aver soddisfatto quanto mi era stato inizialmente richiesto. \\
	Le \gls{Milestone} previste sono infatti state raggiunte nei tempi prestabiliti, con un'eccezione: la presentazione dell'elaborato al team di lavoro, prevista per la fine dell'ottava settimana, è stata anticipata alla conclusione della settima settimana così da evitare l'incorrere di eventuali imprevisti. \\
	Nella prima settimana di lavoro, in aggiunta a quanto pianificato e su indicazione del tutor aziendale, ho studiato e configurato un \gls{Modulo} aggiuntivo dedicato alla ricerca che può essere considerato il passo intermedio tra quanto offerto nativamente dall'ambiente di sviluppo e i motori di ricerca più avanzati analizzati in seguito. Lo studio di tale modulo non ha comunque influito, a livello di tempistiche, su quanto pianificato per le settimane successive alla prima, ritrovandomi dunque allineato con quanto pianificato. \\
	Gli obiettivi aziendali, come già presentato nella \hyperref[subsub:priorita_degli_obiettivi_aziendali]{sezione 2.2.2, in concomitanza della "priorità degli obiettivi aziendali"}, sono stati suddivisi in due differenti tipologie: obiettivi minimi e massimi. \\
	Di seguito, riporterò una tabella nel quale viene indicato lo stato di superamento degli obiettivi dello stage a cui seguirà una specifica più dettagliata.

	\begin{longtable}{| C{2cm} | C{7cm} | C{2cm} |}
		\toprule
		\textbf{ID} & \textbf{Descrizione} & \textbf{Esito} \\ \hline
		\endhead	% Permette di ripetere l'intestazione nelle nuove pagine
		\midrule
		
		%%%%%	INIZIO BLOCCO FUNZIONALITA'	%%%%%
		
		\multicolumn{3}{|c|}{\cellcolor[HTML]{EFEFEF}\textbf{Obiettivi obbligatori (min) }} \\ \hline
		
		%%%%%	INIZIO BLOCCO TEST	%%%%%
		
		min01 &  Analisi dei punti di forza e debolezza dei prodotti \gls{Solr} ed \gls{ElasticSearch} & Soddisfatto \\ \hline
		
		%%%%%	FINE BLOCCO TEST	%%%%%
		
		%%%%%	INIZIO BLOCCO TEST	%%%%%
		
		min02 &  Realizzazione del prototipo in \gls{Drupal} con le funzioni minime di ricerca & Soddisfatto \\ \hline
		
		%%%%%	FINE BLOCCO TEST	%%%%%
		
		%%%%%	FINE BLOCCO FUNZIONALITA'	%%%%%
		
		
		
		%%%%%	INIZIO BLOCCO FUNZIONALITA'	%%%%%
		
		\multicolumn{3}{|c|}{\cellcolor[HTML]{EFEFEF}\textbf{Obiettivi desiderabili e opzionali (max) }} \\ \hline
		
		%%%%%	INIZIO BLOCCO TEST	%%%%%
		
		max01 & Comparazione dei due motori di ricerca esaminati con altri di riferimento nel mercato & Soddisfatto \\ \hline
		
		%%%%%	FINE BLOCCO TEST	%%%%%
		
		%%%%%	INIZIO BLOCCO TEST	%%%%%
		
		max02 &  Indicazioni su possibili interventi sui siti web istituzionali per quanto riguarda la user experience di navigazione, a seguito delle potenzialità espresse dai motori di ricerca & Soddisfatto \\ \hline
		
		%%%%%	FINE BLOCCO TEST	%%%%%
		
		%%%%%	FINE BLOCCO FUNZIONALITA'	%%%%%
		
		\bottomrule
		\caption{Superamento degli obiettivi dello stage}
	\end{longtable}

	\begin{itemize}
		
		\item {\textbf{[min01] Analisi dei punti di forza e debolezza dei prodotti \gls{Solr} ed \gls{ElasticSearch}}: Durante lo stage, ho avuto modo di esplorare le funzionalità offerte da queste due tecnologie, focalizzandomi su quelle di possibile interesse per l'azienda. Un confronto approfondito dei due motori di ricerca non si limiterebbe però alle funzionalità che ho analizzato. Le differenze di maggior rilievo emergono infatti studiandone l'aspetto sistemistico. Personalmente, ho avuto modo di esplorare una piccola parte della visione sistemistica fornita dalle due tecnologie, concentrandomi invece sulle richieste dell'azienda e quindi sullo studio delle funzionalità di possibile interesse per i siti informativi Camerali;}
		
		\item {\textbf{[min02] Realizzazione del prototipo in \gls{Drupal} con le funzioni minime di ricerca}: Per ogni tecnologia di ricerca analizzata ho creato un'istanza \gls{Drupal} ad essa dedicata, attraverso la quale ho verificato l'eventuale presenza delle funzionalità di possibile interesse per i siti informativi Camerali. Ho avuto inoltre modo di creare un prototipo che utilizzasse i dati e le configurazioni dell'attuale sito in produzione della Camera di Commercio di Verona, cambiandone la tecnologia di ricerca precedentemente utilizzata con il fine di migliorarne le funzionalità offerte;}
		
		\item {\textbf{[max01] Comparazione dei due motori di ricerca esaminati con altri di riferimento nel mercato}: Come spiegato nella \hyperref[sec:individuazione_dei_motori_di_ricerca]{Sezione 3.1 : Individuzione dei motori di ricerca}, sono state prese in considerazione ulteriori tecnologie di ricerca oltre a quelle esaminate;}
		
		\item {\textbf{[max02] Indicazioni su possibili interventi sui siti web istituzionali per quanto riguarda la user experience di navigazione, a seguito delle potenzialità espresse dai motori di ricerca}: Come evidenziato nella \hyperref[sub:possibile_evoluzione]{Sezione 3.3.2 : Possibile evoluzione}, ho proposto l'aggiunta di alcune funzionalità agli attuali siti informativi Camerali, prendendo come esempio l'attuale sito in produzione della Camera di Commercio di Verona (\cite{site:vr}), creandone un prototipo che utilizzasse le funzionalità proposte.}
		
	\end{itemize}

	A seguito del mio stage, l'azienda è stata inoltre in grado di ottenere alcuni risultati immediati:
	\begin{itemize}
		\item {La ricerca dei siti informativi delle Camere di Commercio di Bologna, Palermo-Enna e Sicilia Orientale ha visto un'evoluzione della tecnologia di ricerca utilizzata;}
		\item {E' stata migliorata la configurazione della ricerca del sito informativo della Camera di Commercio di Torino.}
	\end{itemize}

	\subsection{Personali}
	Lo stage svolto ha reso possibile la mia introduzione al mondo lavorativo in un'importante azienda che opera nel settore informatico. Nei due mesi trascorsi in \nomeAzienda, ho avuto l'opportunità di confrontarmi con personale qualificato, in possesso di un'ottica diversa da quella a cui sono abituato da studente universitario quale sono. Ho potuto applicare alcune delle conoscenze apprese durante il mio percorso di studi, evolvendo dalla mera conoscenza teorica ad una conoscenza pratica, certamente molto più richiesta e di interesse nell'attuale mercato del lavoro.
	
	\begin{itemize}
		
		\item {Motivazioni professionali: Come già affermato, lo stage in \nomeAzienda mi ha permesso di entrare in contatto con personale esperto e qualificato, che lavora nel mondo dell'informatica da anni, permettendomi così di espandere la mia rete di conoscenze di figure che operano nel mio stesso settore. Inoltre, ho potuto affinare le mie capacità di problem solving ed esplorare tecnologie attuali correlate ad un tema che ritengo di particolare interesse. Tutto ciò ha permesso di aumentare e migliorare le mie conoscenze e competenze, andando ad arricchire il mio curriculum professionale;}
		
		\item {Motivazioni economiche: L'azienda ha rispettato gli accordi inizialmente stabiliti, fornendomi un rimborso spese e buoni pasto;}
		
		\item {Motivazioni personali: Ho avuto modo di valutare se il percorso formativo da me scelto fosse quello corretto e se quanto fatto durante lo stage fosse realmente ciò a cui mi voglio dedicare nella vita.}
		
	\end{itemize}

\section{Conoscenze acquisite}
L'esperienza di stage in \nomeAzienda ha permesso di ampliare le mie conoscenze sia dal punto di vista organizzativo, sia da quello tecnologico. Più in dettaglio:

	\subsubsection{Conoscenze in ambito organizzativo}
	Come previsto inizialmente, l'attività di stage prevedeva incontri quotidiani con il tutor, in modo tale da risolvere eventuali dubbi o decidere di approfondire determinati ambiti di interesse. Così facendo, ho imparato a svolgere le attività programmate senza sprecare tempo su dettagli a volte irrilevanti. Eventuali problemi bloccanti, che non mi permettevano di procedere con il lavoro, sono stati tempestivamente risolti richiedendo l'intervento del tutor aziendale. Il numero di richieste è stato comunque limitato, in modo tale da non compromettere il lavoro e i progetti che il tutor aziendale stava seguendo. Così facendo, ho imparato a lavorare in autonomia, confrontandomi con personale più esperto solamente qualora lo ritenessi necessario.

	\subsubsection{Conoscenze in ambito tecnologico}
	Durante lo stage ho avuto modo di studiare ed esplorare nuove tecnologie; in particolare:
	\begin{itemize}
		\item[--] {ho avuto modo di conoscere e rapportarmi con un \gls{CMSg} (\gls{Drupal});}
		\item[--] {ho avuto modo di confrontarmi con query di un certo grado di complessità, per la comprensione di determinate funzionalità offerte dall'ambiente \gls{Drupal};}
		\item[--] {ho appreso il funzionamento di \gls{Acquia Dev Desktop 2}, rimanendo in ambito locale;}
		\item[--] {ho utilizzato il sistema \gls{Git} mediante un'interfaccia grafica in ambiente Windows;}
		\item[--] {ho avuto modo di conoscere ed utilizzare due tra i maggiori motori di ricerca \gls{open source} attualmente disponibili: \gls{Solr} ed \gls{ElasticSearch}.}
	\end{itemize}

\section{Mondo del lavoro e università a confronto}
L'esperienza di stage offre l'occasione di ampliare le conoscenze per lo più teoriche, apprese durante il percorso di studi, con conoscenze pratiche, ampiamente più richieste nell'attuale mondo del lavoro. Inoltre, rappresenta un'occasione che permette sia all'azienda, sia allo studente, di conoscersi  ed eventualmente di proseguire il rapporto lavorativo anche al termine dello stage. \\
Il ruolo dell'università dovrebbe consistere nel fornire allo studente le conoscenze di base per poter rendersi competitivo nel mercato del lavoro. Questo obiettivo risulta essere particolarmente ambizioso nell'ambito dell'informatica, data la rapida evoluzione delle tecnologie e le numerose aree di interesse che coinvolge. La natura mutevole delle tecnologie può suggerire che il miglior modo per affrontare questa situazione consista nell'impartire, durante i corsi universitari, conoscenze di base facilmente adattabili alle nuove tecnologie emergenti. \\
Focalizzando l'attenzione sull'esperienza di stage che ho svolto presso \nomeAzienda, reputo innanzitutto di fondamentale importanza le conoscenze apprese riguardanti l'organizzazione e i processi aziendali, che permettono allo studente di arrivare preparato in una realtà aziendale particolarmente strutturata, così com'è avvenuto nel mio specifico caso. I progetti e i lavori di gruppo, eventualmente corredati da presentazioni sul lavoro svolto, danno certamente un valore aggiunto al profilo dello studente. \\
Le tecnologie presentate nel corso degli studi universitari hanno reso possibile la facile comprensione delle tecnologie con le quali mi sono confrontato durante lo stage. \\
Al giorno d'oggi, sempre più aziende offrono lavoro in ambito web, esattamente com'è avvenuto nel mio caso. Personalmente, ritengo che uno studente laureato in Informatica, per potersi mettere in gioco nel mercato del lavoro, abbia la necessità di avere le basi necessarie ad affrontare i concetti legati ai \gls{CMSg}, e ai servizi \gls{REST}. Riterrei dunque opportuno approfondire queste tematiche in ambito universitario, organizzando ad esempio seminari che trattino questi argomenti. \\
In conclusione, ritengo l'esperienza di stage uno strumento molto utile e professionalizzante per lo studente, che va molte volte a colmare il gap presente tra il mondo accademico e il mondo del lavoro.		% Valutazione retrospettiva

	%******************************************
	% Parte finale
	%******************************************
	\backmatter
	\printglossaries
	
	\input{sections/bibliografia}
	
\end{document}