% !TEX root = ../../Tesi.tex

%******************************************
%	Resoconto dello stage
%******************************************

\chapter{Resoconto dello stage}
\label{cap:resoconto_dello_stage}

	\section{Individuazione dei motori di ricerca}
	In questa sezione motiverò la selezione di Solr e ElasticSearch come motori di ricerca da studiare.
	%Sono inoltre già presenti moduli che permettono l'integrazione di questi motori con l'ambiente Drupal

	\section{Pianificazione}
	In questa sezione presenterò come è avvenuta la pianificazione dello stage.
	%Ambiente di sviluppo in locale
	
	\section{I siti istituzionali delle Camere di Commercio}

		\subsection{Funzionalità di ricerca attuali}
		Questa sezione presenterà la struttura dei siti camerali attualmente in produzione, ponendo l'accento sugli strumenti messi a disposizione all'utente per ritrovare i contenuti in esso presenti.
		
		\subsection{Possibile evoluzione}
		Questa sezione presenterà una possibile evoluzione dei siti camerali, contenente funzionalità di ricerca attualmente non presenti.

	\section{Ricerca nativa Drupal}

		\subsection{Introduzione a Drupal}
		Qui verrà introdotto Drupal, spiegando cos'è e come funziona.
		
		\subsection{Ricerca di base e avanzata}
		Qui verranno presentate le principali funzionalità offerte dalla prima tipologia di ricerca nativa Drupal.
		
		\subsection{Ricerca con Search API}
		Qui verranno presentate le principali funzionalità offerte dalla seconda tipologia di ricerca nativa Drupal.
		
		\subsection{Considerazioni di Drupal nativo}
		Questa sezione conterrà conclusioni riguardanti le funzionalità di ricerca offerte globalmente dalla ricerca nativa Drupal.

	\section{Ricerca con Solr}

		\subsection{Introduzione a Solr}
		Qui verrà introdotto Solr, spiegando cos'è e come funziona.
		
		\subsection{Principali funzionalità di ricerca}
		Qui verranno presentate le principali funzionalità di ricerca offerte dal motore di ricerca Solr, di possibile interesse per i siti camerali.
			
		\subsection{Integrazione con Drupal}
		Qui verranno discusse le funzionalità di ricerca derivanti dall'integrazione tra Solr e Drupal.
			
			\subsubsection{Apache Solr}
			Qui verranno discusse le funzionalità di ricerca derivanti dall'integrazione tra Solr e Drupal mediante il modulo Apache Solr Search.
			
			\subsubsection{Search API Solr}
			Qui verranno discusse le funzionalità di ricerca derivanti dall'integrazione tra Solr e Drupal mediante il modulo Search API Solr Search.
	
	\section{Ricerca con ElasticSearch}
	
		\subsection{Introduzione a ElasticSearch}
		Qui verrà introdotto ElasticSearch, spiegando cos'è e come funziona.

		\subsection{Principali funzionalità di ricerca}
		Qui verranno presentate le principali funzionalità di ricerca offerte dal motore di ricerca ElasticSearch, di possibile interesse per i siti camerali.
		
		\subsection{Integrazione con Drupal}
		Qui verranno discusse le funzionalità di ricerca derivanti dall'integrazione tra ElasticSearch e Drupal.
		%Accenno alla versione Drupal8?
		
			\subsubsection{Search API ElasticSearch}
			Qui verranno discusse le funzionalità di ricerca derivanti dall'integrazione tra ElasticSearch e Drupal mediante il modulo Search API ElasticSearch.
			
	\section{Considerazioni finali sui motori di ricerca esaminati}
	Questa sezione conterrà un confronto tra le principali funzionalità, possibilmente di interesse per l'azienda, offerte dalle tecnologie esaminate e quale di queste potrebbe essere la più adatta ai siti camerali.