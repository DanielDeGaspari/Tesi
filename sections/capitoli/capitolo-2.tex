% !TEX root = ../../Tesi.tex

%******************************************
%	Lo stage
%******************************************

\chapter{Lo stage}
\label{cap:stage}

\section{Gli stage in azienda}

Le motivazioni che spingono l'azienda ad ospitare stage sono molteplici:

\begin{itemize}
	\item[--] {Una prima motivazione è legata alla continua ricerca di innovazione da portare all'interno dell'azienda. Capita spesso che i dipendenti di \nomeAzienda siano impegnati in parallelo in vari progetti, con scadenze anche stringenti, non avendo tempo dunque da dedicare all'approfondimento di tecnologie che potrebbero portare miglioramenti all'azienda e ai suoi prodotti. Una risorsa come uno stagista universitario risulta dunque essere di gran interesse, permettendo l'esplorazione di nuove tecnologie, senza però dover rallentare altre attività e progetti in opera;}

	\item[--] {Menti derivanti dal mondo universitario hanno le caratteristiche adatte a portare nuovi modi di pensare e di vedere le cose, avendo molte volte la capacità di affrontare i problemi sotto diversi punti di vista rispetto a quelli usualmente trattati aziendalmente. Tutto ciò, assieme all'esperienza che l'azienda può fornire, può evolvere in nuove idee e iniziative per portare un valore aggiunto all'azienda stessa;}
	
	\item[--] {Uno stage che porta ad un soddisfacimento sia dello studente, sia dell'azienda, può far sì che il rapporto tra le due parti continui anche al termine dello stage. In questo modo viene garantito all'azienda un periodo di prova nel quale può decidere se proporre o meno, al termine dello stage, un contratto di assunzione, in modo tale da incrementare o ringiovanire l'organico aziendale.}

\end{itemize}

\section{L'offerta di stage}

	\subsection{Presentazione del progetto}
	Lo scopo dello stage consisteva nell'analisi delle caratteristiche dei motori di ricerca \gls{open source} nell'ambito dei siti web di tipo informativo e, in particolare, lo studio e il confronto delle funzionalità di possibile interesse per i siti web istituzionali delle Camere di Commercio. \\
	In primo luogo era richiesto un approfondimento delle caratteristiche istituzionali dei siti web delle Camere di Commercio. \\
	Successivamente, l'azienda richiedeva lo studio della tecnologia \gls{Drupal} e dell'analisi delle potenzialità e specificità dei motori di ricerca \gls{Solr} e \gls{ElasticSearch}. \\
	Veniva inoltre richiesto un prototipo di un sito web in tecnologia \gls{Drupal}, con i due motori di ricerca precedentemente citati. \\
	Infine, era richiesta una relazione finale delle potenzialità emerse nell'utilizzo dei due motori di ricerca. \\
	
	\subsection{Aspettative aziendali}
	
		\subsubsection{Le milestone}
		Il Piano di Lavoro, redatto assieme al tutor aziendale che mi ha seguito durante lo stage in \nomeAzienda, prevede, tra l'altro, una serie di \gls{Milestone}; ad ognuna di esse, sono associati i prodotti sviluppati entro ogni corrispondente scadenza. Di seguito ne riporto l'elenco completo:
		
		\begin{itemize}
			\item {Fine prima settimana: Ambiente di sviluppo configurato e funzionante (n.b. per ambiente di sviluppo si intende installazione in locale di un sito web \gls{Drupal});}
			\item {Fine seconda settimana: Relazione con approfondimenti riguardanti \gls{Solr};}
			\item {Fine quarta settimana: Realizzazione prototipo di sito web in \gls{Drupal} integrato con \gls{Solr};}
			\item {Fine quinta settimana: Relazione con approfondimenti riguardanti \gls{ElasticSearch};}
			\item {Fine settima settimana: Realizzazione prototipo di sito web in \gls{Drupal} integrato con \gls{ElasticSearch};}
			\item {Fine ottava settimana: Relazione conclusiva e presentazione dell’elaborato al team di lavoro.}		
		\end{itemize}

		\begin{figure}[htbp]
			\begin{center}
				\includegraphics[width=10cm]{milestone}
				\captionof{figure}{Milestone e metodologia di lavoro}
			\end{center}
		\end{figure}
	
		\subsubsection{Prodotti attesi}		
		L'attività di stage prevedeva la produzione di un insieme di prodotti. Di seguito, ne è riportato l'elenco:
		
		\begin{itemize}
			\item {Documento: relazione sul motore di ricerca \gls{Solr} che riporti le caratteristiche funzionali ed architetturali della soluzione, pregi e difetti e possibile utilizzo all'interno del contesto InfoCamere;}
			\item {Documento: relazione sul motore di ricerca \gls{ElasticSearch} che riporti le caratteristiche funzionali ed architetturali della soluzione, pregi e difetti e possibile utilizzo all'interno del contesto InfoCamere;}
			\item {Documento: relazione finale di comparazione dei due motori di ricerca;}
			\item {Software: prototipo di sito web in tecnologia \gls{Drupal} che utilizza motore di ricerca \gls{Solr};}
			\item {Software: prototipo di sito web in tecnologia \gls{Drupal} che utilizza motore di ricerca \gls{ElasticSearch};}
			\item {Documento: relazione conclusiva (slide) sull'esperienza dell'uso dei prototipi, pregi e difetti nell'uso dei motori di ricerca in \gls{Drupal} e comparazioni}.
		\end{itemize}
	
		\subsubsection{Priorità degli obiettivi aziendali}
		Gli obiettivi aziendali riguardanti lo stage possono essere suddivisi in:
		\begin{itemize}
			 \item[--] {obiettivi minimi: vincolanti in quanto richieste primarie del committente;}
			 \item[--] {obiettivi massimi: non vincolanti o strettamente necessari, ma dal riconoscibile valore aggiunto.}
	 	\end{itemize}
 	
		Di seguito, vengono riportati gli obiettivi primari e massimi stabiliti:
		
		\begin{longtable}{| C{2cm} | C{9cm} |}
			\toprule
			\textbf{ID} & \textbf{Descrizione} \\ \hline
			\endhead	% Permette di ripetere l'intestazione nelle nuove pagine
			\midrule
			
			%%%%%	INIZIO BLOCCO FUNZIONALITA'	%%%%%
			
			\multicolumn{2}{|c|}{\cellcolor[HTML]{EFEFEF}\textbf{Obiettivi obbligatori (min) }} \\ \hline
			
				%%%%%	INIZIO BLOCCO TEST	%%%%%
				
				min01 &  Analisi dei punti di forza e debolezza dei prodotti \gls{Solr} ed \gls{ElasticSearch} \\ \hline
				
				%%%%%	FINE BLOCCO TEST	%%%%%
				
				%%%%%	INIZIO BLOCCO TEST	%%%%%
				
				min02 &  Realizzazione del prototipo in \gls{Drupal} con le funzioni minime di ricerca  \\ \hline
				
				%%%%%	FINE BLOCCO TEST	%%%%%

			%%%%%	FINE BLOCCO FUNZIONALITA'	%%%%%



			%%%%%	INIZIO BLOCCO FUNZIONALITA'	%%%%%

			\multicolumn{2}{|c|}{\cellcolor[HTML]{EFEFEF}\textbf{Obiettivi desiderabili e opzionali (max) }} \\ \hline
			
				%%%%%	INIZIO BLOCCO TEST	%%%%%
				
				max01 & Comparazione dei due motori di ricerca esaminati con altri di riferimento nel mercato \\ \hline
				
				%%%%%	FINE BLOCCO TEST	%%%%%
				
				%%%%%	INIZIO BLOCCO TEST	%%%%%
				
				max02 &  Indicazioni su possibili interventi sui siti web istituzionali per quanto riguarda la user experience di navigazione, a seguito delle potenzialità espresse dai motori di ricerca   \\ \hline
				
				%%%%%	FINE BLOCCO TEST	%%%%%
			
			%%%%%	FINE BLOCCO FUNZIONALITA'	%%%%%
			
			\bottomrule
			\caption{Obiettivi dello stage stabiliti da InfoCamere}
		\end{longtable}
		

	\subsection{Vincoli}
	Qui verranno presentati i vincoli legati allo stage.

\section{Vantaggi personali}
In questa sezione descriverò le aspettative personali legate all'accettazione dell'offerta di stage.
