% !TEX root = ../../Tesi.tex

%******************************************
% L'azienda
%******************************************

\chapter{L'azienda}
\label{cap:azienda}

\section{Il Profilo Aziendale}
\label{sec:il_profilo_aziendale}

	\subsection{Le origini: Cerved}
	Nata inizialmente come Cerved (Centro Regionale Veneto Elaborazione Dati), \nomeAzienda è stata fondata nel Dicembre del 1974 a Padova dal Professor Mario Volpato, allora Presidente della Camera di Commercio di Padova e Professore di Calcolo delle probabilità all'Università di Padova. \\
	L'obiettivo era di raccogliere e conservare i dati ufficiali anagrafici e amministrativi delle imprese della provincia di Padova in un modo nuovo rispetto a quanto previsto fino ad allora: la conservazione di quei dati su un registro cartaceo, come si faceva dal medioevo ai tempi delle comunità dei mercanti, non bastava più a garantire l'efficienza del mercato e a stimolare lo sviluppo economico. \\
	Le prime tecnologie informatiche aprivano nuovi orizzonti al trattamento massivo e veloce dei dati. Evolveva rapidamente la concezione di una gestione intelligente delle notizie amministrative sulla vita delle imprese, per trasformarle in informazioni rielaborabili ed utilizzabili in modi nuovi da tutti. Nasceva l’idea di valorizzare i dati ufficiali forniti dalle imprese, restituendoli al mercato e alle imprese stesse come informazioni utili per accrescere la propria competitività e progettare lo sviluppo.\\
	Si gettava il seme dell’efficienza nell’organizzazione delle Camere di Commercio, una base nuova per costruire un patto trasparente e vantaggioso tra imprese e Pubblica Amministrazione.
	
	\subsection{Anni '90: InfoCamere}
	All'inizio degli anni '90 aumentava sempre più la competizione globale e le sfide per portare l'Italia nella modernità. A tal fine, nel 1993, venne emanata una riforma (legge 29 Dicembre 1993, n. 580) che attribuiva alle Camere un'autonomia rispetto al governo centrale, mediante attribuzione della potestà statutaria e di autonomia finanziaria, oltre al riconoscimento del ruolo finalizzato alla pubblicizzazione delle imprese. Le Camere di Commercio Italiane hanno così modo di vedere un profondo rinnovamento in vari ambiti e in particolar modo nell'ambito tecnologico. \\
	Nel 1995, per scissione da Cerved, nasce InfoCamere che raccoglie la sfida di realizzare il Registro delle imprese. Previsto dal codice civile fin dal 1942 e mai attuato, in due anni, con uno di anticipo sulle previsioni, il risultato è raggiunto: prende vita il primo esempio in Europa di registro pubblico sulle imprese  totalmente telematico; assieme ad un ecosistema di servizi sviluppati attorno al Registro delle imprese, è stato possibile semplificare i processi tra le imprese stesse e la Pubblica Amministrazione. \\
	A quella sfida ne seguono altre che rispondono ai nomi di ‘firma digitale’, ‘posta elettronica certificata’, ‘comunicazione unica’, ecc... . \\
	Attraverso InfoCamere, servizi e tecnologie digitali di frontiera diventavano patrimonio quotidiano della comunità delle imprese e dei professionisti, influendo sulle abitudini di lavoro di migliaia di italiani e stimolando i processi di innovazione nella Pubblica Amministrazione.
	
	\begin{figure}[htbp]
		\begin{center}
			\includegraphics[height=3cm]{logo-infocamere}
			\captionof{figure}{Logo infocamere}
		\end{center}
	\end{figure}

	\subsection{Servizi offerti}
	\nomeAzienda è la società consortile di informatica delle Camere di Commercio Italiane. Ha realizzato e gestisce il sistema telematico nazionale che collega tra loro tutte le Camere di Commercio, oltre alle rispettive sedi distaccate. \\
	Sua funzione istituzionale è anche la gestione e divulgazione del patrimonio informativo camerale, con particolare riferimento alle informazioni derivanti dal Registro delle imprese.\\
	Le banche dati camerali sono rese disponibili direttamente a imprese, pubbliche amministrazioni, professionisti e cittadini tramite il portale delle Camere di Commercio. \\
	La Società fornisce alle pubbliche amministrazioni l'accesso al Registro Imprese, assicurando loro l'accessibilità dei dati senza oneri, salvo quelli per la fornitura telematica e i servizi a valore aggiunto. \\
	Tramite il sito del Registro delle imprese si può accedere agli strumenti per lo svolgimento delle pratiche telematiche, tra cui la Comunicazione Unica per l'attività d'impresa, valida anche per Agenzia delle Entrate, INPS, INAIL e Albo Artigiani. Il Registro Imprese è inoltre uno strumento di trasparenza amministrativa che fornisce un contributo importante nella lotta contro la criminalità economica. L'azienda ha infatti sviluppato per le autorità investigative alcuni servizi che, in questa direzione, consentono analisi mirate per monitorare fenomeni anomali. \\
	InfoCamere ha realizzato, per conto delle Camere di Commercio, l'infrastruttura tecnologica che garantisce il corretto funzionamento degli Sportelli Unici per le Attività Produttive (SUAP). \\
	Tra le realizzazioni di InfoCamere per il Sistema camerale vi è anche la procedura informatica che consente di gestire il servizio di conciliazione online (Concilia Camera), fornendo così ad imprese, consumatori e professionisti uno strumento che permette di ricevere assistenza specializzata nel raggiungimento di un accordo per risolvere in modo semplice, rapido, economico e sicuro una controversia, evitando di ricorrere alla giustizia ordinaria. \\
	InfoCamere è inoltre l'Autorità di Certificazione Nazionale che rilascia i certificati digitali delle Carte Tachigrafiche.
	La società si è dotata di un Sistema di Gestione della Sicurezza delle Informazioni certificato secondo lo standard ISO/IEC 27001, avendo conseguito nel 2012 la prima certificazione di conformità ISO/IEC 27001:2005 e a Marzo 2015 la ricertificazione secondo la nuova versione ISO/IEC 27001: 2013.

\section{Organizzazione aziendale}
\label{sec:organizzazione_aziendale}
Questa sezione conterrà l'organizzazione dell'azienda; essendo un'azienda molto strutturata e con molti dipendenti, l'organizzazione aziendale si concentrerà sul contesto in cui ho operato.

\section{Processi aziendali}
\label{sec:processi_aziendali}
Questa sezione parlerà dei processi interni aziendali; essendo un'azienda molto strutturata e con molti dipendenti, i processi descritti si concentreranno sul contesto in cui ho operato.

\section{Tecnologie utilizzate}
\label{sec:tecnologie_utilizzate}
Questa sezione presenterà le tecnologie utilizzate dall'azienda, ristrette all'ambito in cui ho operato.

\section{Rapporto con l'innovazione}
\label{sec:rapporto_con_innovazione}
Questa sezione presenterà il rapporto dell'azienda con l'avanzamento tecnologico.
