% !TEX root = ../../Tesi.tex

%******************************************
%	Valutazione retrospettiva
%******************************************

\chapter{Valutazione retrospettiva}
\label{valutazione_retrospettiva}

\section{Bilancio degli obiettivi}

	\subsection{Aziendali}
	
	Riprendendo gli obiettivi posti inizialmente dall'azienda riguardanti lo stage, presentati nella \hyperref[sub:obiettivi_posti_azienda]{sezione 2.2.2 : Obiettivi posti dall'azienda}, posso affermare di aver soddisfatto quanto mi era stato inizialmente richiesto. \\
	Le \gls{Milestone} previste sono infatti state raggiunte nei tempi prestabiliti, con un'eccezione: la presentazione dell'elaborato al team di lavoro, prevista per la fine dell'ottava settimana, è stata anticipata alla conclusione della settima settimana così da evitare l'incorrere di eventuali imprevisti. \\
	Nella prima settimana di lavoro, in aggiunta a quanto pianificato e su indicazione del tutor aziendale, ho studiato e configurato un \gls{Modulo} aggiuntivo dedicato alla ricerca che può essere considerato il passo intermedio tra quanto offerto nativamente dall'ambiente di sviluppo e i motori di ricerca più avanzati analizzati in seguito. Lo studio di tale modulo non ha comunque influito, a livello di tempistiche, su quanto pianificato per le settimane successive alla prima, ritrovandomi dunque allineato con quanto pianificato. \\
	Gli obiettivi aziendali, come già presentato nella \hyperref[subsub:priorita_degli_obiettivi_aziendali]{sezione 2.2.2, in concomitanza della "priorità degli obiettivi aziendali"}, sono stati suddivisi in due differenti tipologie: obiettivi minimi e massimi. \\
	Di seguito, riporterò una tabella nel quale viene indicato lo stato di superamento degli obiettivi dello stage a cui seguirà una specifica più dettagliata.

	\begin{longtable}{| C{2cm} | C{7cm} | C{2cm} |}
		\toprule
		\textbf{ID} & \textbf{Descrizione} & \textbf{Esito} \\ \hline
		\endhead	% Permette di ripetere l'intestazione nelle nuove pagine
		\midrule
		
		%%%%%	INIZIO BLOCCO FUNZIONALITA'	%%%%%
		
		\multicolumn{3}{|c|}{\cellcolor[HTML]{EFEFEF}\textbf{Obiettivi obbligatori (min) }} \\ \hline
		
		%%%%%	INIZIO BLOCCO TEST	%%%%%
		
		min01 &  Analisi dei punti di forza e debolezza dei prodotti \gls{Solr} ed \gls{ElasticSearch} & Soddisfatto \\ \hline
		
		%%%%%	FINE BLOCCO TEST	%%%%%
		
		%%%%%	INIZIO BLOCCO TEST	%%%%%
		
		min02 &  Realizzazione del prototipo in \gls{Drupal} con le funzioni minime di ricerca & Soddisfatto \\ \hline
		
		%%%%%	FINE BLOCCO TEST	%%%%%
		
		%%%%%	FINE BLOCCO FUNZIONALITA'	%%%%%
		
		
		
		%%%%%	INIZIO BLOCCO FUNZIONALITA'	%%%%%
		
		\multicolumn{3}{|c|}{\cellcolor[HTML]{EFEFEF}\textbf{Obiettivi desiderabili e opzionali (max) }} \\ \hline
		
		%%%%%	INIZIO BLOCCO TEST	%%%%%
		
		max01 & Comparazione dei due motori di ricerca esaminati con altri di riferimento nel mercato & Soddisfatto \\ \hline
		
		%%%%%	FINE BLOCCO TEST	%%%%%
		
		%%%%%	INIZIO BLOCCO TEST	%%%%%
		
		max02 &  Indicazioni su possibili interventi sui siti web istituzionali per quanto riguarda la user experience di navigazione, a seguito delle potenzialità espresse dai motori di ricerca & Soddisfatto \\ \hline
		
		%%%%%	FINE BLOCCO TEST	%%%%%
		
		%%%%%	FINE BLOCCO FUNZIONALITA'	%%%%%
		
		\bottomrule
		\caption{Superamento degli obiettivi dello stage}
	\end{longtable}

	\begin{itemize}
		
		\item {\textbf{[min01] Analisi dei punti di forza e debolezza dei prodotti \gls{Solr} ed \gls{ElasticSearch}}: Durante lo stage, ho avuto modo di esplorare le funzionalità offerte da queste due tecnologie, focalizzandomi su quelle di possibile interesse per l'azienda. Un confronto approfondito dei due motori di ricerca non si limiterebbe però alle funzionalità che ho analizzato. Le differenze di maggior rilievo emergono infatti studiandone l'aspetto sistemistico. Personalmente, ho avuto modo di esplorare una piccola parte della visione sistemistica fornita dalle due tecnologie, concentrandomi invece sulle richieste dell'azienda e quindi sullo studio delle funzionalità di possibile interesse per i siti informativi Camerali;}
		
		\item {\textbf{[min02] Realizzazione del prototipo in \gls{Drupal} con le funzioni minime di ricerca}: Per ogni tecnologia di ricerca analizzata ho creato un'istanza \gls{Drupal} ad essa dedicata, attraverso la quale ho verificato l'eventuale presenza delle funzionalità di possibile interesse per i siti informativi Camerali. Ho avuto inoltre modo di creare un prototipo che utilizzasse i dati e le configurazioni dell'attuale sito in produzione della Camera di Commercio di Verona, cambiandone la tecnologia di ricerca precedentemente utilizzata con il fine di migliorarne le funzionalità offerte;}
		
		\item {\textbf{[max01] Comparazione dei due motori di ricerca esaminati con altri di riferimento nel mercato}: Come spiegato nella \hyperref[sec:individuazione_dei_motori_di_ricerca]{Sezione 3.1 : Individuzione dei motori di ricerca}, sono state prese in considerazione ulteriori tecnologie di ricerca oltre a quelle esaminate;}
		
		\item {\textbf{[max02] Indicazioni su possibili interventi sui siti web istituzionali per quanto riguarda la user experience di navigazione, a seguito delle potenzialità espresse dai motori di ricerca}: Come evidenziato nella \hyperref[sub:possibile_evoluzione]{Sezione 3.3.2 : Possibile evoluzione}, ho proposto l'aggiunta di alcune funzionalità agli attuali siti informativi Camerali, prendendo come esempio l'attuale sito in produzione della Camera di Commercio di Verona (\cite{site:vr}), creandone un prototipo che utilizzasse le funzionalità proposte.}
		
	\end{itemize}

	A seguito del mio stage, l'azienda è stata inoltre in grado di ottenere alcuni risultati immediati:
	\begin{itemize}
		\item {La ricerca dei siti informativi delle Camere di Commercio di Bologna, Palermo-Enna e Sicilia Orientale ha visto un'evoluzione della tecnologia di ricerca utilizzata;}
		\item {E' stata migliorata la configurazione della ricerca del sito informativo della Camera di Commercio di Torino.}
	\end{itemize}

	\subsection{Personali}
	Lo stage svolto ha reso possibile la mia introduzione al mondo lavorativo in un'importante azienda che opera nel settore informatico. Nei due mesi trascorsi in \nomeAzienda, ho avuto l'opportunità di confrontarmi con personale qualificato, in possesso di un'ottica diversa da quella a cui sono abituato da studente universitario quale sono. Ho potuto applicare alcune delle conoscenze apprese durante il mio percorso di studi, evolvendo dalla mera conoscenza teorica ad una conoscenza pratica, certamente molto più richiesta e di interesse nell'attuale mercato del lavoro.
	
	\begin{itemize}
		
		\item {Motivazioni professionali: Come già affermato, lo stage in \nomeAzienda mi ha permesso di entrare in contatto con personale esperto e qualificato, che lavora nel mondo dell'informatica da anni, permettendomi così di espandere la mia rete di conoscenze di figure che operano nel mio stesso settore. Inoltre, ho potuto affinare le mie capacità di problem solving ed esplorare tecnologie attuali correlate ad un tema che ritengo di particolare interesse. Tutto ciò ha permesso di aumentare e migliorare le mie conoscenze e competenze, andando ad arricchire il mio curriculum professionale;}
		
		\item {Motivazioni economiche: L'azienda ha rispettato gli accordi inizialmente stabiliti, fornendomi un rimborso spese e buoni pasto;}
		
		\item {Motivazioni personali: Ho avuto modo di valutare se il percorso formativo da me scelto fosse quello corretto e se quanto fatto durante lo stage fosse realmente ciò a cui mi voglio dedicare nella vita.}
		
	\end{itemize}

\section{Conoscenze acquisite}
L'esperienza di stage in \nomeAzienda ha permesso di ampliare le mie conoscenze sia dal punto di vista organizzativo, sia da quello tecnologico. Più in dettaglio:

	\subsubsection{Conoscenze in ambito organizzativo}
	Come previsto inizialmente, l'attività di stage prevedeva incontri quotidiani con il tutor, in modo tale da risolvere eventuali dubbi o decidere di approfondire determinati ambiti di interesse. Così facendo, ho imparato a svolgere le attività programmate senza sprecare tempo su dettagli a volte irrilevanti. Eventuali problemi bloccanti, che non mi permettevano di procedere con il lavoro, sono stati tempestivamente risolti richiedendo l'intervento del tutor aziendale. Il numero di richieste è stato comunque limitato, in modo tale da non compromettere il lavoro e i progetti che il tutor aziendale stava seguendo. Così facendo, ho imparato a lavorare in autonomia, confrontandomi con personale più esperto solamente qualora lo ritenessi necessario.

	\subsubsection{Conoscenze in ambito tecnologico}
	Durante lo stage ho avuto modo di studiare ed esplorare nuove tecnologie; in particolare:
	\begin{itemize}
		\item[--] {ho avuto modo di conoscere e rapportarmi con un \gls{CMSg} (\gls{Drupal});}
		\item[--] {ho avuto modo di confrontarmi con query di un certo grado di complessità, per la comprensione di determinate funzionalità offerte dall'ambiente \gls{Drupal};}
		\item[--] {ho appreso il funzionamento di \gls{Acquia Dev Desktop 2}, rimanendo in ambito locale;}
		\item[--] {ho utilizzato il sistema \gls{Git} mediante un'interfaccia grafica in ambiente Windows;}
		\item[--] {ho avuto modo di conoscere ed utilizzare due tra i maggiori motori di ricerca \gls{open source} attualmente disponibili: \gls{Solr} ed \gls{ElasticSearch}.}
	\end{itemize}

\section{Mondo del lavoro e università a confronto}
L'esperienza di stage offre l'occasione di ampliare le conoscenze per lo più teoriche, apprese durante il percorso di studi, con conoscenze pratiche, ampiamente più richieste nell'attuale mondo del lavoro. Inoltre, rappresenta un'occasione che permette sia all'azienda, sia allo studente, di conoscersi  ed eventualmente di proseguire il rapporto lavorativo anche al termine dello stage. \\
Il ruolo dell'università dovrebbe consistere nel fornire allo studente le conoscenze di base per poter rendersi competitivo nel mercato del lavoro. Questo obiettivo risulta essere particolarmente ambizioso nell'ambito dell'informatica, data la rapida evoluzione delle tecnologie e le numerose aree di interesse che coinvolge. La natura mutevole delle tecnologie può suggerire che il miglior modo per affrontare questa situazione consista nell'impartire, durante i corsi universitari, conoscenze di base facilmente adattabili alle nuove tecnologie emergenti. \\
Focalizzando l'attenzione sull'esperienza di stage che ho svolto presso \nomeAzienda, reputo innanzitutto di fondamentale importanza le conoscenze apprese riguardanti l'organizzazione e i processi aziendali, che permettono allo studente di arrivare preparato in una realtà aziendale particolarmente strutturata, così com'è avvenuto nel mio specifico caso. I progetti e i lavori di gruppo, eventualmente corredati da presentazioni sul lavoro svolto, danno certamente un valore aggiunto al profilo dello studente. \\
Le tecnologie presentate nel corso degli studi universitari hanno reso possibile la facile comprensione delle tecnologie con le quali mi sono confrontato durante lo stage. \\
Al giorno d'oggi, sempre più aziende offrono lavoro in ambito web, esattamente com'è avvenuto nel mio caso. Personalmente, ritengo che uno studente laureato in Informatica, per potersi mettere in gioco nel mercato del lavoro, abbia la necessità di avere le basi necessarie ad affrontare i concetti legati ai \gls{CMSg}, e ai servizi \gls{REST}. Riterrei dunque opportuno approfondire queste tematiche in ambito universitario, organizzando ad esempio seminari che trattino questi argomenti. \\
In conclusione, ritengo l'esperienza di stage uno strumento molto utile e professionalizzante per lo studente, che va molte volte a colmare il gap presente tra il mondo accademico e il mondo del lavoro.