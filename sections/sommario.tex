% !TEX root = ../Tesi.tex

%******************************************
% Sommario
%******************************************
\cleardoublepage
\phantomsection
\pdfbookmark{Sommario}{Sommario}
\begingroup
\let\clearpage\relax
\let\cleardoublepage\relax
\let\cleardoublepage\relax

\chapter*{Sommario}

Il presente documento descrive il lavoro svolto durante il periodo di stage dal laureando \autore, della durata di circa trecento ore, presso l'azienda \nomeAzienda di \locazioneAzienda. \\
Gli obiettivi da raggiungere erano molteplici. \\
Lo scopo dello stage consisteva nell'analisi delle caratteristiche dei motori di ricerca \gls{open source} nell'ambito dei siti web di tipo informativo.\\
In primo luogo era richiesto un approfondimento delle caratteristiche istituzionali dei siti web delle Camere di Commercio. \\
Successivamente, l'azienda richiedeva di analizzare le potenzialità e specificità dei motori di ricerca \gls{Solr} e \gls{ElasticSearch}. \\
Il passo successivo consisteva nel realizzare un prototipo di un sito web in tecnologia \gls{Drupal}, con i due motori di ricerca precedentemente citati. \\
Infine, era richiesta una relazione finale sulle potenzialità emerse nell'utilizzo dei due motori di ricerca. \\
I primi due capitoli del presente documento hanno lo scopo di presentare il contesto aziendale in cui è stato sostenuto lo stage e di spiegare come il progetto di stage si renda utile all’interno della strategia aziendale. Il terzo capitolo documenta invece lo svolgimento dello stage, descrivendo le attività che sono state portate a termine, i punti salienti del progetto stesso e le principali scelte attuate. Il quarto ed ultimo capitolo presenta infine una valutazione sullo svolgimento dello stage rispetto agli obiettivi aziendali e alle conoscenze acquisite dallo studente.


\endgroup

\vfill