%******************************************
% Acronimi
%******************************************
\renewcommand{\acronymname}{Acronimi e abbreviazioni}

\newacronym[description={\glslink{CMSg}{Content Management System}}]
			{cms}{CMS}{Content Management System}

%******************************************
% Glossario
%******************************************
\newglossaryentry{open source}
{
	name=\glslink{open source}{Open source},
	text=open source,
	sort=open source,
	description={Software di cui i detentori dei diritti rendono pubblico il codice sorgente, permettendo ad altri 		programmatori di apportarvi modifiche. Questo meccanismo è regolato tramite l’applicazione di apposite licenze d’uso}
}

\newglossaryentry{Solr}
{
	name=\glslink{Solr}{Solr},
	text=Sorl,
	sort=solr,
	description={testo test}
}

\newglossaryentry{ElasticSearch}
{
	name=\glslink{ElasticSearch}{ElasticSearch},
	text=ElasticSearch,
	sort=elasticsearch,
	description={testo test}
}

\newglossaryentry{Drupal}
{
	name=\glslink{Drupal}{Drupal},
	text=Drupal,
	sort=drupal,
	description={Drupal è un \gls{CMSg}, rilasciato sotto licenza \gls{open source}, che permette la creazione di siti Internet, blog e portali, gallerie di immagini, forum di discussione, piattaforme intranet e molto altro. Essa è altresì un’applicazione completamente web based e può quindi essere utilizzata attraverso un semplice browser}
}

\newglossaryentry{CMSg}
{
	name=\glslink{cms}{CMS},
	text=Content Management System,
	sort=content management system,
	description={E’ un software per la realizzazione e la gestione di siti dinamici, che possono accrescere e mutare il proprio contenuto continuamente. Un \gls{CMSg} consente al committente del sito di occuparsi direttamente della sua gestione senza intermediari esterni}
}