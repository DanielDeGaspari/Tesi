%%% ****************************************** %%%
%	File contenente impostazioni della tesi	     %
%%% ****************************************** %%%

%%% ****************************************** %%%
%	Informazioni generali						 %
%%% ****************************************** %%%

% ******************************************
%	Autore
% ******************************************
\newcommand{\autore}{Daniel De Gaspari}

% ******************************************
%	Tesi
% ******************************************
\newcommand{\titoloTesi}{Analisi di motori di ricerca Open Source per siti web informativi}
\newcommand{\tipologiaTesi}{Tesi di laurea triennale}

% ******************************************
%	Univeristà
% ******************************************
\newcommand{\universita}{Università degli Studi di Padova}
\newcommand{\cdl}{Corso di Laurea in Informatica}
\newcommand{\dipartimento}{Dipartimento di Matematica "Tullio Levi-Civita"}
\newcommand{\luogo}{Padova}
\newcommand{\myAA}{2016/2017}
\newcommand{\dataDiscussione}{Dicembre 2017}

% ******************************************
%	Relatore
% ******************************************
\newcommand{\statusrelatore}{Prof.}
\newcommand{\relatore}{Tullio Vardanega}

% ******************************************
%	Azienda
% ******************************************
\newcommand{\nomeAzienda}{InfoCamere S.C.p.A. }
\newcommand{\locazioneAzienda}{Padova (PD)}


%%% ****************************************** %%%
%	Impostazioni di graphicx					 %
%%% ****************************************** %%%
\graphicspath{{images/}}									% path a cartella contenente le immagini


%%% ****************************************** %%%
%	Impostazioni di glossaries					 %
%%% ****************************************** %%%
%******************************************
% Acronimi
%******************************************
\renewcommand{\acronymname}{Acronimi e abbreviazioni}

%******************************************
% Glossario
%******************************************
\newglossaryentry{test}
{
	name=\glslink{test}{TEST},
	text=test,
	sort=test,
	description={testo test}
} 										% DataBase dei termini
\makeglossaries
\newcommand{\glsfirstoccur}{\appendix{{[g]}}}


%%% ****************************************** %%%
%	Impostazioni di hyperref					 %
%%% ****************************************** %%%
\hypersetup{
	colorlinks = true,
	linktocpage = true,
	pdfstartpage = 1,
	pdfstartview = FitV,
	breaklinks = true,
	pdfpagemode = UseNone,
	pageanchor = true,
	pdfpagemode = UseOutlines,
	plainpages = false,
	bookmarksnumbered,
	bookmarksopen = true,
	bookmarksopenlevel = 1,
	hypertexnames = true,
	pdfhighlight = /0,
	urlcolor = webbrown,
	linkcolor = blue,
	citecolor = webgreen,
	pdftitle={\titoloTesi}
	pdfauthor={\textcopyright\ \autore, \universita, \cdl},
	pdfsubject={},
	pdfkeywords={},
	pdfcreator={pdfLaTeX},
	pdfproducer={LaTeX}
}


%%% ****************************************** %%%
%	Impostazioni di xcolor						 %
%%% ****************************************** %%%
\definecolor{webgreen}{rgb}{0,.5,0}
\definecolor{webbrown}{rgb}{.6,0,0}
\definecolor{blue}{cmyk}{0.71,0.53,0.00,0.12}


%**************************************************************
% Impostazioni di biblatex
%**************************************************************
\bibliography{bibliografia} % database di biblatex 

\defbibheading{bibliography} {
	\cleardoublepage
	\phantomsection 
	\addcontentsline{toc}{chapter}{\bibname}
	\chapter*{\bibname\markboth{\bibname}{\bibname}}
}

\setlength\bibitemsep{1.5\itemsep} % spazio tra entry

\DeclareBibliographyCategory{opere}
\DeclareBibliographyCategory{web}

%\addtocategory{opere}{womak:lean-thinking}
\addtocategory{web}{site:registro_imprese}

\defbibheading{opere}{\section*{Riferimenti bibliografici}}
\defbibheading{web}{\section*{Siti Web consultati}}



%%% ****************************************** %%%
%	Creazione nuova label per raw				 %
%%% ****************************************** %%%
\newcounter{mycounter}
\newcommand{\mylabel}[1]{\refstepcounter{mycounter} \label{#1}}



%%% ****************************************** %%%
%	Impostazioni di xcolor						 %
%%% ****************************************** %%%
\definecolor{webgreen}{rgb}{0,.5,0}
\definecolor{webbrown}{rgb}{.6,0,0}
\definecolor{blue}{cmyk}{0.71,0.53,0.00,0.12}